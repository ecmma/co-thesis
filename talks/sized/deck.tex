% !TEX program = xelatex
\documentclass[t,aspectratio=169,9pt]{beamer}
\usetheme{gumint}

\usepackage{
  minted,
  fontspec,
  verbatim,
  hyperref
}

\hypersetup{
  colorlinks=false,% hyperlinks will be black
    pdfborderstyle={/S/U/W 0.1}, % underline links instead of boxes
    linkbordercolor=GumintPrimary,       % color of internal links
    citebordercolor=GumintPrimary,     % color of links to bibliography
    filebordercolor=GumintPrimary,   % color of file links
    urlbordercolor=GumintPrimary        % color of external links
}

\setmonofont{PragmataPro Mono Liga}
\setminted{fontsize=\footnotesize}
\usemintedstyle{gumint}

\title{MiniAgda}
\subtitle{Terminazione e produttività con \textit{sized types}}
\date{}
\author{Edoardo Marangoni}
\email{ecmm@anche.no}

\begin{document}

\frame[plain]{\titlepage}

\begin{frame}[t,plain]
	\frametitle{Agenda}
	\tableofcontents
\end{frame}


\section[intro]{Introduzione}
\subsection[totalità]{Totalità}
\begin{frame}[fragile]{Totalità}
	Nei proof assistant, la totalità è necessaria per mantenere la
	\textbf{consistenza} del sistema: se, infatti, ammettiamo definizioni ricorsive
	non terminanti, si possono dimostrare teoremi falsi.
	\vfill
\begin{minted}{coq}
Parameter bad : nat -> nat.
Axiom bad_ax : forall x, bad x = 1 + (bad x).

Theorem n_neq_sn : forall (x:nat), (x = S x) -> False.
Proof. intros.  induction x.
  + inversion H.
  + inversion H.  apply IHx in H1. assumption.
Qed.

Theorem zero_eq_one: 0 = 1.
  assert (bad 0 = 1 + (bad 0)).
  + apply bad_ax.
  + destruct bad; simpl in H.
    * assumption.
    * apply n_neq_sn in H. exfalso. assumption.
Qed.
\end{minted}
\end{frame}

\begin{frame}[fragile]{Totalità}
	Oltre ai motivi di consistenza logica, in un linguaggio con tipi dipendenti la
		{\bf totalità} è necessaria anche per assicurare la terminazione della fase di
	type-checking.

	\begin{minted}{agda}
 fooIsTrue : foo == true
 fooIsTrue = refl
\end{minted}

	Per mostrare che \texttt{refl} sia una dimostrazione del fatto che \texttt{foo
		== true}, Agda impiega il type checker per mostrare che \texttt{foo} sia
	effettivamente ``riscrivibile'' in \texttt{true}; se \texttt{foo} non termina,
	allora nemmeno il type-checker termina. In questa esposizione ci concentriamo su
		{\bf terminazione} e {\bf produttività}, ossia due criteri per decidere se
	permettere delle definizioni di funzione ricorsive e co-ricorsive in linguaggi
	totali.
\end{frame}

\section[induttivi]{Tipi induttivi}
\subsection{Introduzione}

\begin{frame}[fragile]{Introduzione}
	Numeri naturali (induttivi) in Coq
	\begin{minted}{coq}
Inductive Nat : Type := 
  | zero : Nat 
  | succ : Nat -> Nat

(* ricorsione primitiva *)
Fixpoint monus (x y : Nat) : Nat := 
  match x with 
  | zero => zero 
  | succ x' => match y with 
               | zero => succ x'
               | succ y' => monus x' y' 
               end
  end.

(* termina? *)
Fixpoint div (x y : Nat) :  Nat := z
match x with 
| zero => zero
| succ x' => succ (div (monus x y) y)
end.
  \end{minted}
\end{frame}

\begin{frame}[fragile]{Introduzione}
	...e in Agda
	\begin{minted}{agda}
data Nat : Set where 
  zero : Nat 
  succ : Nat -> Nat 

-- ricorsione primitiva
monus : Nat -> Nat -> Nat 
monus zero _ = zero
monus (succ x) zero = succ x
monus (succ x) (succ y) = monus x y

-- termina?
div : Nat -> Nat -> Nat 
div zero _ = zero 
div (succ x) y = succ (div (monus x y) y)
\end{minted}
	Le definizioni di \texttt{div} ovviamente terminano: vengono accettate?
\end{frame}

\subsection{Terminazione}

\begin{frame}[fragile,plain,standout]
	\begin{center}
		\color{GumintPrimary}\Huge \textbf{No!}
	\end{center}
	Coq:
	\begin{minted}{coq}
Cannot guess decreasing argument of fix.
\end{minted}

	Agda:
	\begin{minted}{agda}
Termination checking failed for the following functions:
  div
Problematic calls:
  div (monus x y) y
\end{minted}

	Coq e Agda basano il controllo della terminazione sulla sintassi e non sono
	pertanto in grado di catturare informazioni semantiche rilevanti come, per
	esempio, il fatto che per ogni argomento $x$ e $y$, $monus ~ x ~ y \leq x$:
	pertanto, la definizione di $div$ non può essere accettata.
	\vfill
\end{frame}

\begin{frame}[fragile]{Terminazione (in Agda)}
	Non tutte le funzioni ricorsive sono permesse: \href{https://wiki.portal.chalmers.se/agda/ReferenceManual/Totality\#Terminationchecking}{Agda}, per esempio,
	accetta solo quegli schemi ricorsivi che riesce a dimostrare
	terminanti.
	\begin{itemize}
		\item {{\bf Ricorsione primitiva:}
		      un argomento di una chiamata ricorsiva deve essere
			      {\it esattamente} più piccola ``di un costruttore''.
		      \begin{minted}{agda}
plus : Nat -> Nat -> Nat
plus zero    m = m
plus (suc n) m = suc (plus n m)
\end{minted}
		      }
		      \item{{\bf Ricorsione strutturale:}
		                  Un argomento di una chiamata ricorsiva deve essere una
			                  {\it sottoespressione}.
		                  \begin{minted}{agda}
fib : Nat -> Nat
fib zero          = zero
fib (suc zero)    = suc zero
fib (suc (suc n)) = plus (fib n) (fib (suc n))
		      \end{minted}
		            }
	\end{itemize}

\end{frame}

\section[induttivi sized]{Tipi induttivi sized}
\subsection{Sized-types: intuizione}
\begin{frame}{Sized-types: intuizione}
	Sono un approccio {\bf tipato}: annotiamo i tipi con una {\it size} (o
		{\it taglia}) che esprime un'informazione sulla dimensione dei valori di
	quel tipo. MiniAgda è ``la prima implementazione matura di un sistema con
	sized-types''. \href{https://arxiv.org/abs/1012.4896}{[ref]}

	\vskip15pt

	{
	\center
	``The {\bf untyped} approaches have some shortcomings, including
	the sensitivity of the termination checker to syntactical
	reformulations of the programs, and a lack of means to propagate
	size information through function calls.''
	\href{https://arxiv.org/abs/1012.4896}{[A. Abel, MiniAgda]}
	}

	\vskip15pt

	Per quanto riguarda la terminazione, l'idea di base è la seguente:
	\begin{enumerate}
		\item {Associamo una {\it taglia} $i$ ad ogni tipo induttivo $D$;}
		\item {Controlliamo che la taglia diminuisca nelle chiamate ricorsive.}
	\end{enumerate}
\end{frame}
\subsection{Implementazione}
\begin{frame}{Implementazione}
	\begin{itemize}
		\item{
		L'{\bf altezza} di un elemento induttivo $d \in N$ è il
			{\it numero di costruttori} di $d$. Possiamo immaginare $d$ come
		un albero in cui i nodi sono i costruttori: per esempio,
		l'altezza di un numero naturale $n$ è $n+1$ (perché c'è il
		costruttore $zero$ come foglia con $n$ costruttori $succ$);
		l'altezza di una lista di lunghezza $n$ è $n+1$ (perché c'è
		il costruttore $nil$).
		}
		\item {
		      Dato un certo tipo induttivo $N$ associamo ad esso una {\bf
				      taglia} $i$, ottenendo un tipo $N^i$ che contiene solo quei
		      $d$ la cui altezza {\bf è minore} di $i$, ossia ogni $d \in
			      N^i$ ha un'altezza che ha come {\bf upper bound} la taglia
		      $i$. In altre parole, dato un elemento induttivo $d \in
			      N^{i}$, sappiamo che possiamo ``smontare'' $d$ nei suoi
		      costruttori al più $i-1$ volte. }
		\item  {
		      In un linguaggio d.t. possiamo modellare la taglia come un
		      tipo $Size$ e i {\bf sized-types} come membri di $Size
			      \rightarrow Set$ (nel caso più semplice senza polimorfismo
		      del tipo sottostante). Definiamo, inoltre, la funzione {\it
				      successore}\footnote{nell'articolo di Abel la funzione
			      successore è indicata col simbolo $\$$. Ci atteniamo alle
			      definizioni della libreria di Agda e usiamo $\uparrow$.} sulle
		      taglie $\uparrow : Size \rightarrow Size$ tale per cui
		      $(\uparrow i) > i$.
		      }
	\end{itemize}
\end{frame}
\begin{frame}[fragile]{Implementazione}
	Definiamo i numeri naturali {\it sized} come membri di $Size \rightarrow Set$.
	\begin{minted}{agda}
-- Naturali *sized*
--         |--------| parametro implicito
--         v        v
data Nat : {i : Size} -> Set where
    zero : {i : Size} -> Nat {↑ i} 
    succ : {i : Size} -> Nat {i} -> Nat {↑ i}
\end{minted}
	Il costruttore $zero$ ha tipo $\forall i ~ SNat^{\uparrow i}$ perché la taglia è
	un {\it upper bound}: anche se interpretiamo il costruttore \texttt{zero} come un albero
	con un singolo nodo (pertanto con altezza $1$), vogliamo modellare il fatto che
	la sua taglia sia $i \geq 1$.
\end{frame}
\begin{frame}{Implementazione}
	\begin{itemize}
		\item {{\bf Sottotipi}
		      Per quanto appena detto, viene naturale per i tipi induttivi la regola
		      di sottotipizzazione $N^i \leq N^{\uparrow i} \leq \cdots \leq
			      N^{\omega}$ dove $N^{\omega}$ è un elemento la cui altezza è ignota
		      (nella sintassi di Agda, la {\it taglia ignota} è $\infty$).
		      }
		\item {{\bf Dot patterns}
		      Definendo alcune funzioni (come \texttt{monus} nella prossima slide),
		      per mantenere la linearità della parte sinistra del pattern-match è
		      necessario utilizzare i {\it dot patterns}, usati quando l'unico valore ben
		      tipato per un argomento è definito da un pattern per un altro argomento,
		      distinguendo quindi tra patterni ordinari e pattern {\it inaccessibili}.
		      }
	\end{itemize}
\end{frame}
\begin{frame}[fragile]{Implementazione}
	\begin{minted}{agda}
monus : {i : Size} -> Nat {i} -> Nat {∞} -> Nat {i}
monus .{↑ j} (zero {j}) n = zero {j}
monus .{↑ j} (succ {j} x) (zero {∞}) = succ {j} x
-- anche se `monus x y` ha una taglia strettamente minore di i, il tipo del
-- risultato è SNat i (e questo caso, `monus x y`, è ben tipato per
-- sottotipizzazione); in altre parole, l'informazione che in questo caso il
-- risultato ha taglia strettamente minore di x è persa. 
monus .{↑ j} (succ {j} x) (succ {∞} y) = monus {j} x y

-- Divisione euclidea *sized*
div : {i : Size} -> Nat {i} -> Nat {∞} -> Nat {i}
div .{↑ j} (zero {j}) y = zero {j}
div .{↑ j} (succ {j} x) y = succ (div {j} (monus {j} x y) y)
\end{minted}
	Nella definizione di $div$, siccome $x$ ha taglia $j < i$, lo stesso vale anche
	per $monus~x~y$: la chiamata ricorsiva a $div$ avviene sulla taglia $j$
	strettamente minore di $i$; questo permette di dimostrare la terminazione della
	funzione.
\end{frame}
\begin{frame}[fragile]{Esempio più complesso}
	\begin{minted}{agda}
data List (A : Set) : Set where
  nil : List A 
  cons : A -> List A -> List A

mapList : {A : Set} -> {B : Set} -> (A -> B) -> List A -> List B
mapList f nil = nil 
mapList f (cons a x) = cons (f a) (mapList f x)

data Rose (A : Set) : Size → Set where
  rose : {i : Size} -> A -> List (Rose A i) → Rose A (↑ i)

mapRose : {A : Set} -> {B : Set} -> (A -> B) -> {i : Size} -> Rose A i -> Rose B i
mapRose f (rose a rs) = rose (f a) (mapList (mapRose f) rs)
\end{minted}
	Senza {\it sized-types}:
	\begin{minted}{agda}
Termination checking failed for the following functions:
  mapRose
Problematic calls:
  mapRose f
\end{minted}
\end{frame}
\section[coinduttivi]{Tipi co-induttivi}
\subsection{Introduzione}
\begin{frame}[fragile]{Introduzione}
	Nonostante sia necessaria la totalità, possiamo lo stesso ``osservare'' oggetti
	infiniti tramite la {\bf co-induzione}, concetto duale all'induzione. L'esempio
	principale dei tipi co-induttivi è quello degli {\bf stream}. Esempio basilare in
	Coq
	\begin{minted}{coq}
CoInductive NatStream : Type := cons { hd: nat; tail: NatStream }.
CoFixpoint countFrom (n:nat) := cons nat n (countFrom (n+1)). (* =*> (cons n (cons n+1 (cons n+2 (...)))) *)
\end{minted}
	...e in Agda
	\begin{minted}{agda}
record NatStream : Set where
  coinductive
  constructor _::_
  field
    head : Nat
    tail : NatStream

open NatStream public 

countFrom : Nat -> NatStream 
head (countFrom x) = x 
tail (countFrom x) = countFrom (x + 1)
\end{minted}
\end{frame}
\subsection{Guardedness}
\begin{frame}[fragile]{Guardedness}
	Non possiamo richiedere la {\it terminazione} di una funzione come
	\texttt{countFrom} (che, per definizione, non termina);
	invece, richiediamo la {\bf produttività}: \textit{``[...]
	we require productivity, which means that the next portion can always be
	produced in finite time. A simple criterion for productivity which can be
	checked syntactically is guardedness..''}
	\href{https://arxiv.org/pdf/1012.4896.pdf}{[ref]}

	\vfill

	In Agda, il controllo della produttività di una funzione co-ricorsiva è basato
	sulla {\it guardedness} della definizione, ossia richiediamo che la definizione di
	una funzione (la parte destra) sia un (co-)costruttore e che ogni chiamata
	ricorsiva sia direttamente ``sotto'' esso. Questa definizione è accettata per
	guardedness:
	\begin{minted}{agda}
repeat : Nat -> NatStream
head (repeat x) = x 
tail (repeat x) = repeat x
\end{minted}
\end{frame}
\begin{frame}[fragile]{Guardedness}
	La seguente dipende da $F$:

	\begin{minted}{agda}
repeatF : Nat -> (NatStream -> NatStream) -> NatStream
head (repeatF x _) = x 
tail (repeatF x f) = f (repeatF x f)
\end{minted}

	se $F$ è, esempio, $tail$, la definizione si riduce a sé stessa dopo una ricorsione:

	\begin{minted}{agda}
tail (repeat a) -> tail (a :: tail repeat a) -> tail repeat a -> ...
\end{minted}
	se invece $F$ mantiene la lunghezza dello stream o la incrementa,
	\texttt{repeatF} è produttiva; i controlli puramente sintattici, però, non
	possono catturare questo aspetto.

	(Nota: tornando agli esempi precedenti, il controllo sulla terminazione della
	funzione \texttt{div} fallisce perché il controllo sintattico non riesce a
	``capire'' che \texttt{monus} diminuisce (o lascia intatto) il parametro
	\texttt{x}; qui accade l'opposto: il controllo fallisce se non riesce a
	``capire'' che `F` incrementa lo stream o lo lascia intatto.)
\end{frame}
\section[coinduttivi sized]{Tipi coinduttivi sized}
\subsection{Implementazione}
\begin{frame}[fragile]{Implementazione}
	\begin{itemize}
		\item{{\bf Profondità}
		            La \textbf{profondità} di un elemento coinduttivo $d \in D$ è il
		            \textit{numero di co-costruttori} di $d$. Uno stream interamente
		            costruito avrà profondità $\omega$.
		      }
		\item{{\bf Taglia}
		Dato un certo tipo co-induttivo $C$ associamo ad esso una taglia $i$,
		ottenendo un tipo $C^i$ che contiene solo quei $d$ la cui profondità {\bf
		è maggiore} di $i$, ossia ogni $d \in C^i$ ha una profondità che ha come
			{\bf lower bound} la taglia $i$. In altre parole, dato un elemento
		co-induttivo $d \in C^{i}$, sappiamo che possiamo ``osservare'' $d$ nei
		suoi costruttori almeno $i$ volte.
		}
	\end{itemize}

	\begin{minted}{agda}
record NatStream {i : Size} : Set where 
  coinductive
  field 
    head : Nat 
    --          |-----| j di tipo `Size` minore di `i`
    --          v     v
    tail : {j : Size< i} -> NatStream {j}
  \end{minted}
\end{frame}
\begin{frame}[fragile]{Implementazione}
	La funzione \texttt{countFrom} con i sized-types:
	\begin{minted}{agda}
countFrom : {j : Size} -> Nat -> NatStream {j}
head (countFrom {_} x) = x
tail (countFrom {j} x) {i}  = countFrom {i} (x + 1)
\end{minted}
	\begin{itemize}
		\item{{\bf Sottotipi}
		            Dualmente, se per i tipi induttivi vale $I^i \leq
			            I^{\uparrow i} \leq \cdots \leq I^{\omega}$, per i
		            tipi co-induttivi vale l'opposto: $C^i \geq
			            C^{\uparrow i} \geq \cdots \geq C^{\omega}$ dove
		            $C^{\omega}$ è un elemento la cui profondità è
		            ignota.
		      }
	\end{itemize}

	Possiamo quindi definire \texttt{repeatF} come segue:
	\begin{minted}{agda}
repeatF : {i : Size}  -> Nat -> ({j : Size< i} -> NatStream {j} -> NatStream {j})  -> NatStream {i}
head (repeatF n _) = n 
tail (repeatF n f) = f (repeatF n f)
\end{minted}
	in questo modo assicuriamo al type checker che la funzione \texttt{f} non
	diminuisce la profondità del suo argomento.
\end{frame}
\end{document}
