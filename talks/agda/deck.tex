% !TEX program = xelatex
\documentclass[t,aspectratio=169,9pt]{beamer}
\usetheme{gumint}

\usepackage{
  minted,
  fontspec,
  verbatim,
  hyperref
}

\hypersetup{
  colorlinks=false,% hyperlinks will be black
    pdfborderstyle={/S/U/W 0.1}, % underline links instead of boxes
    linkbordercolor=GumintPrimary,       % color of internal links
    citebordercolor=GumintPrimary,     % color of links to bibliography
    filebordercolor=GumintPrimary,   % color of file links
    urlbordercolor=GumintPrimary        % color of external links
}

\setbeamertemplate{footline}[frame number]{}
\setmonofont{PragmataPro Mono Liga}
\setminted{fontsize=\footnotesize}
\usemintedstyle{gumint}

\title{A brief introduction to Agda}
\subtitle{}%\it or, how I stopped worrying and love the typechecker }
\date{\today}
\author{Edoardo Marangoni}
\email{ecmm@anche.no}

\begin{document}

\frame[plain]{\titlepage}

\begin{frame}[t,plain]
	\frametitle{Agenda}
	\tableofcontents
\end{frame}

\section[intro]{Introduction}
\subsection[hist]{History}
\begin{frame}{History}
  \vfill
  \begin{itemize}
    \item {V1.0 (approx. 1999) by C. Coquand}
    \item {V2.0 (approx. 2007) is a complete rewrite by U. Norell}
    \item {draws inspiration from previous works such as Alf, Alfa, Cayenne,
      Coq, ... , Automath }
  \end{itemize}
  \vfill
\end{frame}
\subsection[what?]{What is it?}
\begin{frame}[fragile]{What is it?}
  \begin{itemize}
    \item {a {\bf total} dependently typed programming language}
    \item {an extension of intuitionistic Martin-Löf type-theory}
    \item {a proof assistant (due to its dependent typing and Curry-Howard's
      correspondence)}
  \end{itemize}
  \vfill
The anatomy of a simple Agda definition:
\begin{minted}{agda}

open import Data.Bool
open import Data.String

-- Define a data type for greetings
data Greeting : Set where           -- Set is somewhat close to Coq's Type
 hello : Greeting   -- no name given
 -- this underscore is for prefix, infix, postfix, mixfix arguments to
 -- constructors
 --    ▼
 hello,_ : String → Greeting -- greet a person

greet : String → Greeting --               see?
--                                           ▼
greet x = if x == "" then hello else (hello, x)
\end{minted}
  \vfill
\end{frame}

\section[induction]{Induction in Agda}
\subsection[induction]{Basics}
\begin{frame}[fragile]{Induction}
  Let's start from the basics: red-black trees. Ok, no, natural numbers:
\begin{minted}{agda}
data Nat : Set where
 zero : Nat
 suc : Nat → Nat
\end{minted}
  In Agda, data structures need not be tagged with ``inductive'' or
  ``coinductive'' (albeit records do).
  Let's see a simple example of a function:
\begin{minted}{agda}
_+_ : Nat → Nat → Nat
_+_ = ? -- <- This is a goal. It allows interactive development.
\end{minted}
\end{frame}
\begin{frame}[noframenumbering, fragile]{Induction}
  Let's start from the basics: red-black trees. Ok, no, natural numbers:
\begin{minted}{agda}
data Nat : Set where
 zero : Nat
 suc : Nat → Nat
\end{minted}
  In Agda, data structures need not be tagged with ``inductive'' or
  ``coinductive'' (albeit records do).
  Let's see a simple example of a function:
\begin{minted}{agda}
_+_ : Nat → Nat → Nat
x + x₁ = ? --  <- Using holes, we can _split_ data types; in this case,
           -- Agda gives us the two arguments x and x₁.
\end{minted}
\end{frame}
\begin{frame}[noframenumbering, fragile]{Induction}
  Let's start from the basics: red-black trees. Ok, no, natural numbers:
\begin{minted}{agda}
data Nat : Set where
 zero : Nat
 suc : Nat → Nat
\end{minted}
  In Agda, data structures need not be tagged with ``inductive'' or
  ``coinductive'' (albeit records do).
  Let's see a simple example of a function:
\begin{minted}{agda}
_+_ : Nat → Nat → Nat
-- Let's split again on x:          (this is Agda's pattern-match)
zero + x₁ = {! !}
suc x + x₁ = {! !}
\end{minted}
\end{frame}
\begin{frame}[noframenumbering, fragile]{Induction}
  Let's start from the basics: red-black trees. Ok, no, natural numbers:
\begin{minted}{agda}
data Nat : Set where
 zero : Nat
 suc : Nat → Nat
\end{minted}
  In Agda, data structures need not be tagged with ``inductive'' or
  ``coinductive'' (albeit records do).
  Let's see a simple example of a function:
\begin{minted}{agda}
_+_ : Nat → Nat → Nat
zero + x₁ = x₁
suc x + x₁ = suc (x + x₁)
\end{minted}
  Sweet, but how do we prove things?
\end{frame}

\subsection[induction]{Inductive proofs}
\begin{frame}[fragile]{Inductive proofs}
  Let's prove the (left) identity of \texttt{+}.
\begin{minted}{agda}
+-id : ∀ (x : Nat) → ?0 -- <- Yes, we can use holes in type signatures as well!
+-id x = ?1
\end{minted}
What should the type be?

This is the internal definition of (propositional, intensional, ``definitional'') equality.
\begin{minted}{agda}
--        ╭ implicit argument with implicit type
--        │  ╭ implicit argument of type 'Set a' - notice the dependent typing
--        │  │           ╭ read this as 'for all x of type A'
--        ▼  ▼---------▼ ▼-----▼
data _≡_ {a} {A : Set a} (x : A) : A → Set a where
  refl : x ≡ x

cong : ∀ {a b} {A : Set a}  {B : Set b} (f : A → B) {m  n}  → m ≡ n → f m ≡ f n
cong f  refl  = refl
\end{minted}
Two terms in agda are said to be definitionally equal when they both have the
same normal form up to αη-conversion.
\end{frame}

\begin{frame}[fragile]{Inductive proofs}
  Let's prove the (left) identity of \texttt{+}.
\begin{minted}{agda}
+-id : ∀ (x : Nat) → zero + x ≡ x
+-id zero = ? -- <- must be a term of type '(zero + zero) ≡ zero'
+-id (suc x) = ? -- <- must be a term of type  '(zero + suc x) ≡ suc x'
\end{minted}
\end{frame}
\begin{frame}[noframenumbering, fragile]{Inductive proofs}
  Let's prove the (left) identity of \texttt{+}.
\begin{minted}{agda}
+-id : ∀ (x : Nat) → zero + x ≡ x
+-id zero = refl
+-id (suc x) = ? -- <- must be a term of type  '(zero + suc x) ≡ suc x'
\end{minted}
\end{frame}
\begin{frame}[noframenumbering, fragile]{Inductive proofs}
  Let's prove the (left) identity of \texttt{+}.
\begin{minted}{agda}
+-id : ∀ (x : Nat) → zero + x ≡ x
+-id zero = refl
+-id (suc x) = cong suc (+-id x)
\end{minted}
\end{frame}
\begin{frame}[noframenumbering, fragile]{Inductive proofs}
  Let's prove the (left) identity of \texttt{+}.
\begin{minted}{agda}
+-id : ∀ (x : Nat) → zero + x ≡ x
+-id zero = refl
+-id (suc x) = cong suc (+-id x)
\end{minted}
of course, Agda is powerful enough to let this proof be more concise:
\begin{minted}{agda}
+-id : ∀ (x : Nat) → zero + x ≡ x
+-id x = refl
\end{minted}
\end{frame}
\begin{frame}[fragile]{Inductive proofs}
  Let's prove something slightly more complicated: the associativity of
  \texttt{+}. We need to {\it reason} about equality. From Agda's
  \texttt{stdlib}:
\begin{minted}{agda}
--                ╭ notice that `y` is implicit!
--                ▼
_≡⟨⟩_ : ∀ {A} (x {y} : A) → x ≡ y → x ≡ y
_ ≡⟨⟩ x≡y = x≡y

_≡⟨_⟩_ : ∀ (x {y z} : A)  → x ≡ y → y ≡ z → x ≡ z
x ≡⟨ x≡y ⟩ y≡z = trans x≡y y≡z
\end{minted}
\begin{minted}{agda}
+-assoc : ∀ (x y z : Nat) -> (x + y) + z ≡ x + (y + z)
+-assoc zero y z = refl
+-assoc (suc x) y z = {! !} -- Goal: ((suc x + y) + z) ≡ (suc x + (y + z))
\end{minted}
\end{frame}
\begin{frame}[fragile]{Inductive proofs}
  Let's prove something slightly more complicated: the associativity of
  \texttt{+}. We need to {\it reason} about equality. From Agda's
  \texttt{stdlib}:
\begin{minted}{agda}
--                ╭ notice that `y` is implicit!
--                ▼
_≡⟨⟩_ : ∀ {A} (x {y} : A) → x ≡ y → x ≡ y
_ ≡⟨⟩ x≡y = x≡y

_≡⟨_⟩_ : ∀ (x {y z} : A)  → x ≡ y → y ≡ z → x ≡ z
x ≡⟨ x≡y ⟩ y≡z = trans x≡y y≡z
\end{minted}
\begin{minted}{agda}
+-assoc : ∀ (x y z : Nat) -> (x + y) + z ≡ x + (y + z)
+-assoc zero y z = refl
+-assoc (suc x) y z =
  ((suc x) + y) + z
  ≡⟨⟩
  (suc (x + y)) + z
  ≡⟨⟩
  suc ((x + y) + z)
  ≡⟨ ? ⟩ -- Goal :  suc ((x + y) + z) ≡ y'
   ? -- Goal : y' ≡ (suc x + (y + z))
\end{minted}
\end{frame}
\begin{frame}[fragile]{Inductive proofs}
  Let's prove something slightly more complicated: the associativity of
  \texttt{+}. We need to {\it reason} about equality. From Agda's
  \texttt{stdlib}:
\begin{minted}{agda}
--                ╭ notice that `y` is implicit!
--                ▼
_≡⟨⟩_ : ∀ {A} (x {y} : A) → x ≡ y → x ≡ y
_ ≡⟨⟩ x≡y = x≡y

_≡⟨_⟩_ : ∀ (x {y z} : A)  → x ≡ y → y ≡ z → x ≡ z
x ≡⟨ x≡y ⟩ y≡z = trans x≡y y≡z
\end{minted}
\begin{minted}{agda}
+-assoc : ∀ (x y z : Nat) -> (x + y) + z ≡ x + (y + z)
+-assoc zero y z = refl
+-assoc (suc x) y z =
  ((suc x) + y) + z
  ≡⟨⟩
  (suc (x + y)) + z
  ≡⟨⟩
  suc ((x + y) + z)
  ≡⟨ cong suc (+-assoc x y z) ⟩
   ? -- Goal : suc (x + (y + z)) ≡ (suc x + (y + z))
\end{minted}
\end{frame}
\begin{frame}[fragile]{Inductive proofs}
  Let's prove something slightly more complicated: the associativity of
  \texttt{+}. We need to {\it reason} about equality. From Agda's
  \texttt{stdlib}:
\begin{minted}{agda}
--                ╭ notice that `y` is implicit!
--                ▼
_≡⟨⟩_ : ∀ {A} (x {y} : A) → x ≡ y → x ≡ y
_ ≡⟨⟩ x≡y = x≡y

_≡⟨_⟩_ : ∀ (x {y z} : A)  → x ≡ y → y ≡ z → x ≡ z
x ≡⟨ x≡y ⟩ y≡z = trans x≡y y≡z
\end{minted}
\begin{minted}{agda}
+-assoc : ∀ (x y z : Nat) -> (x + y) + z ≡ x + (y + z)
+-assoc zero y z = refl
+-assoc (suc x) y z =
  ((suc x) + y) + z
  ≡⟨⟩
  (suc (x + y)) + z
  ≡⟨⟩
  suc ((x + y) + z)
  ≡⟨ cong suc (+-assoc x y z) ⟩
   suc (x + (y + z))
  ≡⟨⟩
    suc x + (y + z)
  ∎ -- <- reflexivity on equality: _∎ : ∀ (x : A) → x ≡ x
\end{minted}
\end{frame}
\end{document}
